\subsection{Javljanje}

\subsubsection{Javljanje na \v salteru}

\subsubsection{Javljanje putem interneta}

\subsubsection{Javljanje kod savetnika}