\subsection{Redovno javljanje}

Redovno javljanje predstavlja postupak dolaska nezaposlenog lica u Nacionalnu slu\v zbu radi vo\dj enja redovne evidencije o teku\' cem stanju. Nezaposleno lice je u obavezi da se na svaka 3 meseca javi u Nacionalnoj slu\v zbi, i da izvesti zaposlene koji posreduju u zapo\v sljavanju o svom trenutnom statusu.\\

U zavisnosti od nivoa izve\v stavanja nezaposleno lice mo\v ze da bira na\v cin redovnog javljanja, i to:
\begin{itemize}
	\item ukoliko nema potrebu za dodatnim uslugama Nacionalne slu\v zbe, bira jedno od naredna dva:
	\begin{itemize}
		\item Javljanje na \v salteru (slu\v caj upotrebe \ref{su: javljanje na salteru}), ili
		\item Javljanje putem interneta (slu\v caj upotrebe \ref{su: javljanje putem interneta}).
	\end{itemize}
	
	\item ukoliko ima potrebu za dodatnim uslugama Nacionalne slu\v zbe, onda bira Javljanje kod savetnika (slu\v caj upotrebe \ref{su: javljanje kod savetnika}).
\end{itemize}

Unapre\dj enje trenutnog re\v senja predstavlja Javljanje putem interneta. Ovim slu\v cajem upotrebe se uvodi novi deo onlajn sistema Nacionalne slu\v zbe, radi olak\v savanja procesa Redovno Javljanje nezaposlenih lica koji nemaju dodatne potrebe za uslugama Nacionalne slu\v zbe. Doprinosi ovog unapre\dj enja su:
\begin{itemize}
	\item smanjenje reda \v cekanja u Nacionalnoj slu\v zbi,
	\item usmeravanje rada \v salterskog slu\v zbenika na druge (zna\v cajnije) radne aktivnosti i zadatke, i
	\item obavljanje procesa Redovno javljanje iz komfornosti doma.
\end{itemize}

\subsubsection{Javljanje na \v salteru}
\label{su: javljanje na salteru}

\noindent U\v cesnici: Nezaposleno lice (u daljem tekstu NL), \v salterski slu\v zbenik (u daljem tekstu \v SS)
\\
\\ Preduslovi: NL ima evidentacioni karton. \v SS je ulogovan na sistem. 
\\
\\ Postuslovi: Ili je uspe\v sno zabele\v zeno javljanje NL-a ili je NL obave\v sten o svom stanju.
\\ 
\\ Glavni tok:
\begin{enumerate}
	\item NL prilazi automatu i bira opciju ''Redovno javljanje''.
	\item Automat bele\v zi da je opcija ''Redovno javljanje'' odabrana, ra\v cuna naredni broj u redu \v cekanja za tu opciju, i \v stampa papir sa brojem.
	\item NL uzima broj i \v ceka svoj red.
	\item Kada do\dj e na red, NL prilazi \v salteru, zahteva od \v SS da prijavi njegov dolazak, i predaje \v SS-u svoj evidentacioni karton.
	\item \v SS pronalazi NL-a u sistemu.
	\item \v SS unosi u sistem da je NL do\v sao na redovno javljanje.
	\item \v SS zakazuje slede\' ce javljanje i vra\' ca NL-u evidentacioni karton.
\end{enumerate}

\noindent Alternativni tokovi: 
\begin{description}
	\item[A1. Pad sistema] ~\\
	Ukoliko se u bilo kom koraku Glavnog toka dogodi pad sistema na kojem radi \v SS, \v SS ponovo pokre\'ce sistem i prijavljuje se na njega. Prelazi se na korak 5 Glavnog toka.
	
	\item[A2. Status NL-a je ''zamrznut''] ~\\
	Ukoliko u koraku 5 Glavnog toka sistem prika\v ze da je status NL-a ''zamrznut'', \v SS obave\v stava NL-a o njegovom statusu, i slu\v caj upotrebe se zavr\v sava.
\end{description}

\subsubsection{Javljanje putem interneta}
\label{su: javljanje putem interneta}

\subsubsection{Javljanje kod savetnika}
\label{su: javljanje kod savetnika}