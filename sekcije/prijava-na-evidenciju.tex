\section{Slu\v cajevi upotrebe}

\subsection{Prijava na evidenciju}


\subsubsection{Slu\v caj upotrebe: Prijava}

\noindent U\v cesnici: Nezaposleno lice (u daljem tekstu NL), \v salterski slu\v zbenik (u daljem tekstu \v SS)
\\
\\ Preduslovi: NL ima li\v cnu kartu i dokaz o stru\v cnoj spremi. Automat za izdavanje broja je u funkciji. \v SS je ulogovan na sistem. 
\\
\\ Postuslovi: NL je uspe\v sno prijavljen na evidenciju i izdat je evidencioni karton tra\v zioca zaposlenja.
\\
\\ Glavni tok:
\\ NL prilazi automatu i bira opciju "Prijava na evidenciju", uzima broj i \v ceka svoj red. Kada do\dj e na red, prilazi \v salteru i zahteva od \v SS da ga prijavi na evidenciju. \v SS otvara formular za prijavu. \v SS tra\v zi od NL da mu preda li\v cnu kartu i dokaz o stru\v cnoj spremi. NL predaje potrebna dokumenta. \v SS popunjava formular na osnovu predatih dokumenata. Zatim, \v SS daje NL-u da popuni Obrazac za prijavu na evidenciju. NL vra\v ca popunjen obrazac. \v SS popunjava Evidencioni karton za tra\v zioca zaposlenja. Zatim zakazuje prvo javljanje savetniku za zapo\v sljavanje, i postavlja status NL-a na ''aktivan''. \v SS vra\v ca dokumenta NL-u i daje mu evidentacioni karton. Prelazi se na slu\v caj upotrebe \ref{su: prvo javljanje savetniku}.
\\
\\ Alternativni tok: 
\\ Ukoliko se u bilo kom trenutku dogodi pad sistema, \v SS ponovo pokre\'ce sistem, loguje se i popunjava formular ukoliko je sistem pao pre ispe\v snog \v cuvanja.


\subsubsection{Slu\v caj upotrebe: Prvo javljanje savetniku}
\label{su: prvo javljanje savetniku}

\noindent U\v cesnici: Nezaposleno lice (u daljem tekstu NL), savetnik za zapo\v sljavanje (SZ), Administrator sistema (u daljem tekstu A)
\\
\\ Preduslovi: NL ima evidentacioni karton. SZ je ulogovan na sistem. 
\\
\\ Postuslovi: Uspe\v sno je zabele\v zeno javljanje.
\\ 
\\ Glavni tok:
\\ NL dolazi kod SZ-a u kancelariju i predaje mu svoj evidentacioni karton. SZ pronalazi NL-a u sistemu. SZ i NL razgovaraju o NL-ovoj stru\v cnoj spremi i kakvi poslovi ga zanimaju. SZ unosi nove informacije u sistem. Zatim, razgovaraju o ve\v stinama NL-a i SZ unosi nove informacije. SZ tra\v zi informacije o prethodnim zaposlenjima NL-a, ako postoje i te informacije unosi u sistem. Na kraju, SZ zakazuje slede\' ce javljanje i vra\' ca NL-u evidentacioni karton.
\\
\\ Alternativni tok:
\\ Ukoliko NL ne do\dj e pre zakazanog termina, automatski se \v salje zahtev za izmenom podataka o NL-u u sistemu, i to: status NL-a se prebacuje na ''zamrznut'', i traje 6 meseci, nakon \v cega se ponovo vra\' ca na status ''aktivan''.
