\subsection{Kori\v s\' cenje onlajn sistema Nacionalne slu\v zbe}

Onlajn sistem Nacionalne slu\v zbe je veb aplikacija koja omogu\' cava obavljanje odre\dj enih poslovnih funkcija vezanih za eksterne korisnike putem interneta. Kori\v s\' cenjem onlajn sistema, eksterni korisnici mogu uz manje napora i br\v ze da dobiju usluge Nacionalne slu\v zbe.\\

Onlajn sistem prepoznaje dve vrste eksternih korisnika: \textit{nezaposleno lice}, i \textit{predstavnik kompanije sa kojom Nacionalna slu\v zba sara\dj uje} (radi jednostavnosti, ovaj subjekat \'cemo nazivati \textit{kompanija}). Na Slici \ref{dsu: koriscenje onlajn sistema nacionalne sluzbe} prikazani su navedeni korisnici i procesi u kojima u\v cestvuju, a koje \'cemo opisati u daljem tekstu.\\

Za oba korisnika je omogu\' ceno pravljenje naloga i prijavljivanje na sistem. Pravljenje naloga za nezaposlena lica i kompanije je u osnovi sli\v cno. Razlike izme\dj u ova dva procesa su: dodatne aktivnosti koje Nacionalna slu\v zba mora da preuzme pri otvaranju naloga za kompanije i informacije od zna\v caja (formular u korisni\v ckom interfejsu). Zbog toga \' cemo prvo opisati Pravljenje naloga (slu\v caj upotrebe \ref{su: pravljenje naloga}), a zatim opisati pojedinosti pravljenja naloga za kompanije, \v sto podrazumeva Odobrenje naloga za kompanije (slu\v caj upotrebe \ref{su: odobrenje naloga za kompanije}). Slu\v cajem upotrebe \ref{su: prijavljivanje na sistem} detaljno je opisan proces prijavljivanja na sistem. Na kraju prikazujemo procese Pretraga oglasa poslova (slu\v caj upotrebe \ref{su: pretraga oglasa poslova}) i Otvaranje oglasa za novu ponudu za posao (slu\v caj upotrebe \ref{su: otvaranje oglasa za novu ponudu za posao}). Za svaki opisan proces dajemo i izgled korisni\v ckog interfejsa.

\begin{figure}[H]
	\centering
	
	\caption{Dijagram slu\v cajeva upotrebe procesa ''Kori\v s\' cenje onlajn sistema Nacionalne slu\v zbe''.}
	\label{dsu: koriscenje onlajn sistema nacionalne sluzbe}
\end{figure}

\subsubsection{Pravljenje naloga}
\label{su: pravljenje naloga}

\noindent U\v cesnici: Eksterni korisnik (u daljem tekstu EK)
\\
\\ Preduslovi: Ako je tip EK ''Nezaposleno lice'', onda EK ima evidentacioni karton.
\\
\\ Postuslovi: Ili je EK uspe\v sno napravio nalog ili je stanje ostalo nepromenjeno.
\\ 
\\ Glavni tok:
\begin{enumerate}
	\item EK pristupa onlajn sistemu Nacionalne slu\v zbe.
	\item Sistem prikazuje po\v cetnu stranu \ref{}.
	\item EK bira opciju ''Registracija novog korisnika''.
	\item U zavisnosti od tipa EK, Sistem prikazuje odgovaraju\' ci formular.
	\item EK popunjava formular.
	\item EK bira opciju ''Sa\v cuvaj i nastavi''.
	\item Sistem, na osnovu unetih podataka, proverava da li ve\'c postoji takav korisnik.
	\item Sistem ustanovljava da postoji takav korisnik.
	\begin{enumerate}
		\item Sistem prikazuje poruku sa obave\v stenjem i nudi EK opciju da promeni \v lozinku (ukoliko ju je zaboravio).
		\item EK bira da li \v zeli da promeni lozinku.
		\item EK bira da promeni lozinku.
		\begin{enumerate}
			\item Sistem tra\v zi od EK da unese email adresu.
			\item NK unosi email adresu i potvr\dj uje unos.
			\item Sistem obave\v stava EK da mu je poslata nova lozinka na email adresu.
			\item Slu\v caj upotrebe se zavr\v sava.
		\end{enumerate}
		\item EK bira da ne promeni lozinku, pa se prelazi na korak 2.
	\end{enumerate}
	\item Sistem ustanovljava da ne postoji takav korisnik, pa se prelazi na korak 10.
	\item Sistem utvr\dj uje tip EK.
	\item Sistem je utvrdio da je tip EK ''Nezaposleno lice'', pa se prelazi na korak 13.
	\item Sistem je utvrdio da je tip EK ''Kompanija''.
	\begin{enumerate}
		\item Prelazi se na slu\v caj upotrebe \ref{su: odobrenje naloga za kompanije}.
	\end{enumerate}
	\item Sistem otvara novi nalog sa unetim podacima, a zatim \v salje poruku na unetu email adresu za potvr\dj ivanje.
	\item Sistem obave\v stava EK da je nalog uspe\v sno otvoren i da treba da potvrdi otvaranje naloga na email-u.
	\item Kada EK potvrdi otvaranje naloga na email-u, sistem mu prikazuje obave\v stenje da je otvaranje naloga kompletirano.
\end{enumerate}

\noindent Alternativni tokovi: 
\begin{description}
	\item[A1. Nedostupnost sistema] ~\\
	Ukoliko u bilo kom od koraka 1--4 Glavnog toka ne do\dj e do prikaza korisni\v ckog interfejsa (na primer, zbog sporog interneta), EK mo\v ze da sa\v ceka, pa da poku\v sa ponovo, ili da odustane od pravljenja naloga.
	
	\item[A2. Neispravnost unetih podataka] ~\\
	Ukoliko u koraku 6 Glavnog toka sistem utvrdi da neki podatak nije validan ili neko obavezno polje nije popunjeno, sistem generi\v se poruku o gre\v sci i zahteva od EK da ispravno popuni formu. Izvr\v savanje se nastavlja u koraku 5 Glavnog toka.
\end{description}

\subsubsection{Odobrenje naloga za kompanije}
\label{su: odobrenje naloga za kompanije}

\subsubsection{Prijavljivanje na sistem}
\label{su: prijavljivanje na sistem}

\subsubsection{Pretraga oglasa poslova}
\label{su: pretraga oglasa poslova}

\subsubsection{Otvaranje oglasa za novu ponudu za posao}
\label{su: otvaranje oglasa za novu ponudu za posao}

\newpage