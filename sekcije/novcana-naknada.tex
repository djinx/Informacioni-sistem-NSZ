\subsection{Nov\v cana naknada}

Nezaposleno lice dolazi u Nacionalnu slu\v zbu za zapo\v sljavanje radi ostvarivanja prava na nov\v canu naknadu. Kako bi nezaposleno lice ostvarilo pravo na nov\v canu naknadu ono mora da se prvo prijavi za ostvarivanje iste.
\\
Nakon \v sto se prijavi za ostvarivanje prava na nov\v canu naknadu, nezaposleno lice ide kod upravnog referenta da mu preda potrebnu dokumentaciju.
\\ Predata dokumenta prosle\dj uju se zajedno sa prijavom pravnom licu koje na osnovu zakonskih odredbi i priložene dokumentacije odlu\v cuje da li je nezaposleno lice ostvarilo pravo na nov\v canu naknadu ili nije.
\\
Odluka se prosle\dj uje kod direktora na overu i zatim u pisarnicu odakle se \v salje po\v stom nezaposlenom licu.

\subsubsection{Prijava za ostvarivanja prava na nov\v canu naknadu}

\noindent Učesnici: Nezaposleno lice (u daljem tekstu NL), šalterski službenik (u daljem tekstu ŠS)
\\
\\ Preduslovi: NL ima ličnu kartu.
\\
\\ Postuslovi: Prijava je popunjena i prosleđena upravnom referentu.
\\
\\ Glavni tok:
\begin{enumerate}
\item NL prilazi automatu i bira opciju ''Prijava za novčanu naknadu''.
\item Automat bele\v zi da je opcija ''Prijava za novčanu naknadu'' odabrana, ra\v cuna naredni broj u redu \v cekanja za tu opciju, i \v stampa papir sa brojem.
\item NL uzima broj i \v ceka svoj red.
	\item Kada do\dj e na red, NL prilazi evidencionom pultu i zahteva od \v SS da ga prijavi za novčanu naknadu.
	\item \v SS otvara formular za prijavu.
	\item \v SS tra\v zi od NL da mu preda li\v cnu kartu.
	\item NL predaje li\v cnu kartu.
    \item \v SS pronalazi NL u sistemu.
	\item \v SS daje NL-u da popuni Zahtev za ostvarivanje prava na novčanu naknadu.
	\item NL popunjava Zahtev za ostvarivanje prava na novčanu naknadu, a zatim ga vra\v ca SS-u.
	\item \v SS vra\v ca ličnu kartu NL-u i upućuje ga kod upravnog referenta.
	\item Prelazi se na slu\v caj upotrebe \ref{su: prvo javljanje savetniku}.
\end{enumerate}

\begin{description}
	\item[A1. Nezaposleno lice nije prijavljeno na evidenciju] ~\\
	Ukoliko u koraku 8. Glavnog toka ŠS ne može da nađe NL u sistemu, zaključuje da NL nije prijavljeno na evidenciju tržišta rada i prelazi se na slučaj upotrebe \ref{su: prijava}.
\end{description}

\subsubsection{Kompletiranje dokumentacije}
\label{su: kompletiranje dokumentacije}

\noindent Učesnici: Nezaposleno lice (u daljem tekstu NL), upravni referent (u daljem tekstu UR).
\\
\\ Preduslovi: NL je popunio Zahtev za ostvarivanje prava na novčanu naknadu.
\\
\\ Postuslovi: UR je kompletirao dokumentaciju i prosledio je pravnom licu.
\\
\\ Glavni tok:
\begin{enumerate}
\item NL dolazi kod UR-a u kancelariju i predaje mu Zahtev za ostvarivanje prava na novčanu naknadu.
\item UR pronalazi NL u sistemu i dopunjava zahtev podacima iz evidencije.
\item UR traži od NL-a da mu preda potrebnu dokumentaciju.
\item NL-e predaje UR-u potrebnu dokumentaciju.
\item UR preuzima dokumentaciju i zahtev i šalje pravnom licu na tumačenje.
\item Prelazi se na slu\v caj upotrebe \ref{su: resenje}.
\end{enumerate}

\subsubsection{Donošenje rešenja o novčanoj naknadi}
\label{su: resenje}

\noindent Učesnici: Pravno lice.
\\
\\ Preduslovi: Upravni referent je predao Zahtev za ostvarivanje prava na novčanu naknadu i potrebnu dokumentaciju nezaposlenog lica pravnom licu.
\\
\\ Postuslovi: Nezaposleno lice je ili dobilo pravo na novčanu naknadu ili je odbijeno.
\\
\\ Glavni tok:
\begin{enumerate}
\item Pravno lice prima Zahtev za ostvarivanje prava na novčanu naknadu i potrebnu dokumentaciju.
\item Pravno lice proverava da li je sve ispravno popunjeno.
\item \begin{enumerate}
\item Pravno lice tumači zakon i na osnovu zakona i potrebne dokumentacije odobrava novčanu naknadu nezaposlenom licu.
\item Pravno lice tumači zakon i na osnovu zakona i potrebne dokumentacije ne odobrava novčanu naknadu nezaposlenom licu.
\end{enumerate}
\item Pravno lice prosleđuje rešenje o novčanoj naknadi direktoru na overu.
\item Prelazi se na slu\v caj upotrebe \ref{su: gde je pecat}.
\end{enumerate}

\begin{description}
\item [A1. Nezaposleno lice nije predalo svu potrebnu dokumentaciju] ~\\
	U koliko u koraku 2. Glavnog toka primeti da fali neophodna dokumentacija pismenim putem obaveštava nezaposleno lice da donese upravnom referentu nedostajuću dokumentaciju. Prelazi se na slučaj upotrebe \ref{su: kompletiranje dokumentacije}.
\end{description}


\subsubsection{Zaključivanje rešenja}
\label{su: gde je pecat}

\noindent Učesnici: direktor.
\\
\\ Preduslovi: Pravno lice je donelo rešenje o novčanoj naknadi i predalo ga direktoru.
\\
\\ Postuslovi: Rešenje o novčanoj naknadi je overeno.
\\
\\ Glavni tok:
\begin{enumerate}
\item Direktor preuzima rešenje o novčanoj naknadi.
\item Direktor analizira rešenje o novčanoj naknadi.
\item Direktor overava rešenje o novčanoj naknadi i prosleđuje ga pisarnici.
\item Prelazi se na slu\v caj upotrebe \ref{su: pisarnica}.
\end{enumerate}


\subsubsection{Slanje rešenja o novčanoj naknadi nezaposlenom licu}
\label{su: pisarnica}

\noindent Učesnici: Referent prijema i ekspedicije pošte (u daljem tekstu RPEP).
\\
\\ Preduslovi: Direktor je overio rešenje o novčanoj naknadi i prosledio ga pisarnici.
\\
\\ Postuslovi: Rešenje o novčanoj naknadi je poslato nezaposlenom licu.
\\
\\ Glavni tok:
\begin{enumerate}
\item RPEP preuzima overeno rešenje o novčanoj naknadi.
\item RPEP pakuje rešenje u kovertu.
\item RPEP čita adresu nezaposlenog lica iz sistema.
\item RPEP šalje overeno rešenje o novčanoj naknadi na pročitanu adresu.
\end{enumerate}

