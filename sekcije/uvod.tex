\section{Uvod}

Ovaj rad se bavi modeliranjem dela sistema Nacionalne slu\v zbe za zapo\v sljavanje republike Srbije. U okviru ovog rada, obra\' cena je pa\v znja i na mogu\' ca unapre\dj enja sistema.\\

Rad je izra\dj en kao grupni studentski projekat na Matemati\v ckom fakultetu, na studijskom programu Informatika, prve godine Master studija. Projekat je odra\dj en pod nadzorom profesora dr Sa\v se Malkova, u okviru predmeta Informacioni sistemi.\\

U nastavku \' cemo detaljnije opisati rad Nacionalne slu\v zbe.

\subsection{Nacionalna slu\v zba za zapo\v sljavanje}

Nacionalna slu\v zba za zapo\v sljavanje obavlja poslove zapo\v sljavanja, osiguranja za slu\v caj nezaposlenosti, ostvarivanje prava iz osiguranja za slu\v caj nezaposlenosti i druga prava u skladu sa zakonom, odnosno poslove vo\dj enja evidencija u oblasti zapo\v sljavanja, kao i stru\v cno-organizacione, upravne, ekonomsko-finansijske i druge op\v ste poslove u oblasti zapo\v sljavanja i osiguranja za slu\v caj nezaposlenosti, u skladu sa zakonom, svojim statutom i drugim aktima Nacionalne slu\v zbe.

\subsection{Metodologija rada}

Prilikom izrade rada, informacije o Nacionalnoj slu\v zbi su prikupljene iz dva glavna izvora. Prvi izvor predstavljaju zvani\v cna dokumenta Nacionalne slu\v zbe, i to ``Program rada Nacionalne slu\v zbe za zapo\v sljavanje (za 2016. godinu)'' i ``Statut Nacionalne slu\v zbe za zapo\v sljavanje'', \v cijim razmatranjem smo dolazili do formalnih informacija o sistemu. Drugi izvor informacija je razgovor sa zaposlenim licima, \v sto nam je prevashodno omogu\' cilo da steknemo uvid u mogu\' ca unapre\dj enja sistema.\\

Deo sistema koji je analiziran je modeliran kori\v s\' cenjem dijagrama slu\v cajeva upotrebe (engl. \textit{Use Case Diagram}).

\newpage