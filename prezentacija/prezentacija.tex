\documentclass[11pt]{beamer}
\usetheme{Warsaw}

\setbeamercolor*{palette primary}{use=structure,fg=white,bg=violet!60!gray}
\setbeamercolor*{palette quaternary}{fg=violet,bg=white!70!gray}
\setbeamercolor*{item}{fg=violet}
\setbeamertemplate{navigation symbols}{}
\setbeamertemplate{caption}[numbered]

\usepackage[serbian]{babel}
\usepackage[utf8]{inputenc}
\usepackage[]{babel}
\usepackage{amsmath}
\usepackage{amsfonts}
\usepackage{amssymb}
\usepackage{graphicx}

\author{Ajzenhamer Nikola \\ Bukurov Anja \\ Stankovi\' c Vojislav }


\title{Informacioni sistem Nacionalne slu\v zbe za zapo\v sljavanje}

\begin{document}
	
	\begin{frame}
		\titlepage
	\end{frame}


\section{Nacionalna slu\v zba za zapo\v sljavanje}

\begin{frame}{Opis}
Nacionalna slu\v zba za zapo\v sljavanje obavlja poslove 
	\begin{itemize}
		\item zapo\v sljavanja
		\item osiguranja za slu\v caj nezaposlenosti
		\item ostvarivanje prava iz osiguranja za slu\v caj nezaposlenosti i druga prava u skladu sa zakonom
		\item vo\dj enja evidencija u oblasti zapo\v sljavanja
		\item kao i stru\v cno-organi\-zacione, upravne, ekonomsko-finansijske i druge op\v ste poslove u oblasti zapo\v sljavanja i osiguranja za slu\v caj nezaposlenosti, u skladu sa zakonom, svojim statutom i drugim aktima Nacionalne slu\v zbe.
	\end{itemize}
	
\end{frame}

\section{Metodologija rada}
\begin{frame}{Metodologija rada}
	\begin{itemize}
		\item Prilikom izrade rada, informacije o Nacionalnoj slu\v zbi su prikupljene iz dva glavna izvora.
		\item  Prvi izvor predstavljaju zvani\v cna dokumenta Nacionalne slu\v zbe, i to ''Program rada Nacionalne slu\v zbe za zapo\v sljavanje (za 2016. godinu)'' i ''Statut Nacionalne slu\v zbe za zapo\v sljavanje'', \v cijim razmatranjem smo dolazili do formalnih informacija o sistemu. 
		\item Drugi izvor informacija je razgovor sa zaposlenim licima, \v sto nam je prevashodno omogu\' cilo da steknemo uvid u mogu\' ca unapre\dj enja sistema.
	\end{itemize}
\end{frame}

\subsection{Dijagrami}
\begin{frame}{Kori\v s\' ceni dijagrami}
	\begin{itemize}
		\item Zahtevi sistema su modelirani dijagramima slu\v cajeva upotrebe.
		\item BPMN dijagrame (posebno, BPMN dijagrame saradnji) koristili smo za opisivanje interakcije izme\dj u entiteta tokom slu\v cajeva upotrebe.
		\item Dijagrami sekvence koristili smo pri slu\v cajevima upotrebe za koje smatramo da je potrebno dodatno opisati tok scenarija u vremenu.
	\end{itemize}
\end{frame}

\begin{frame}
	\begin{itemize}
		\item Dijagram stanja
		\item Dijagrami toka podataka
		\item Dijagram klasa
	\end{itemize}
\end{frame}

\subsection{Softver}
\begin{frame}
	\begin{itemize}
		\item Kao softversko re\v senje za izradu pomenutih dijagrama, kori\v s\' cen je Visual Paradigm 14.
		\item Kao pristup izrade prototipa korisni\v ckog interfejsa odabran je HTML prototip.
	\end{itemize}
\end{frame}

\section{Opisani slu\v cajevi upotrebe}

\begin{frame}{Prijava na evidenciju}
	\begin{itemize}
		\item Nezaposleno lice dolazi li\v cno u Nacionalnu slu\v zbu za zapo\v sljavanje radi prijave na evidenciju. U zavisnosti od toga da li je lice ve\' c bilo na evidenciji ili ne razlikujemo dva slu\v caja:
		\begin{itemize}
			\item ukoliko lice nije bilo nikada ranije prijavljeno na evideniciji, izvr\v sava se slu\v caj upotrebe Prva prijava.
			\item ukoliko je lice bilo prijavljeno na evidenciju ali je izgubilo status aktivnog tra\v zenja posla zbog neredovnog javljanja izvr\v sava se slu\v caj upotrebe Ponovna prijava.
		\end{itemize}
	\item Nakon prijave, Nezaposleno lice odlazi na Prvo javljanje kod savetnika.
	\end{itemize}
\end{frame}

\begin{frame}{Kori\v s\' cenje onlajn sistema Nacionalne slu\v zbe}
	\begin{itemize}
		\item Onlajn sistem Nacionalne slu\v zbe je veb aplikacija koja omogu\' cava obavljanje odre\dj enih poslovnih funkcija vezanih za eksterne korisnike putem interneta. 
		\item Kori\v s\' cenjem onlajn sistema, eksterni korisnici mogu uz manje napora i br\v ze da dobiju usluge Nacionalne slu\v zbe.
		
		\item Onlajn sistem prepoznaje dve vrste eksternih korisnika: \textit{nezaposleno lice}, i \textit{predstavnik kompanije sa kojom Nacionalna slu\v zba sara\dj uje}.
		
	\end{itemize}
\end{frame}

\begin{frame}{Kori\v s\' cenje onlajn sistema Nacionalne slu\v zbe}
	\begin{itemize}
		
		\item Za oba korisnika prvi korak je Pravljenje naloga. 
		\item U slu\v caju pravljenja naloga za kompaniju, opisan je dodatan slu\v caj upotrebe Odobrenje naloga za kompanije
		\item Slede\' ci korak je Prijavljivanje na sistem
		\item Na kraju prikazujemo procese Pretraga oglasa poslova i Otvaranje oglasa za novu ponudu za posao 
		\item Za svaki opisan proces dajemo i izgled korisni\v ckog interfejsa. 
	\end{itemize}
\end{frame}

\begin{frame}{Redovno javljanje}
	\begin{itemize}
		\item Redovno javljanje predstavlja postupak vo\dj enja redovne evidencije teku\' ceg stanja nezaposlenog lica u Nacionalnoj slu\v zbi.
		\item Nezaposleno lice je u obavezi da se na svaka 3 meseca javi u Nacionalnoj slu\v zbi, i da izvesti zaposlene koji posreduju u zapo\v sljavanju o svom trenutnom statusu.
	\end{itemize}
\end{frame}

\begin{frame}{Redovno javljanje}
	U zavisnosti od nivoa izve\v stavanja nezaposleno lice mo\v ze da bira na\v cin redovnog javljanja, i to:
	\begin{itemize}
		\item ukoliko nema potrebu za dodatnim uslugama Nacionalne slu\v zbe, bira jedno od naredna dva:
		\begin{itemize}
			\item Javljanje putem interneta, ili
			\item Javljanje na \v salteru.
		\end{itemize}
		
		\item ukoliko ima potrebu za dodatnim uslugama Nacionalne slu\v zbe, onda bira Javljanje kod savetnika.
	\end{itemize}
\end{frame}

\begin{frame}{Posredovanje u zapo\v sljavanju}
	\begin{itemize}
		\item Poslodavac \v zeli da zaposli radnika preko NSZ. On dolazi u slu\v zbu i na \v salteru mu ka\v zu da moraju da ga registruju u njihovu bazu poslodavaca. \item Nakon registracije, poslodavac mo\v ze da objavljuje oglase za nepopunjena radna mesta u njegovoj kompaniji.
		\item Nakon toga poslodavac zapo\v sljava radnika i pravi se izve\v staj o realizaciji konkursa.
	\end{itemize}
	
\end{frame}

\begin{frame}{Nov\v cana naknada}
	\begin{itemize}
		\item Nezaposleno lice dolazi u Nacionalnu slu\v zbu za zapo\v sljavanje radi ostvarivanja prava na nov\v canu naknadu. 
		\item Prvi korak je Prijava za ostvarivanja prava na nov\v canu naknadu.
		\item Nakon toga odlazi kod upravnog referenta i obavlja se Kompletiranje dokumentacije.
	\end{itemize}
\end{frame}

\begin{frame}{Nov\v cana naknada}
\begin{itemize}		
		\item Zatim, pravno lice koje na osnovu zakonskih odredbi i prilo\v zene dokumentacije Dono\v si re\v senje o nov\v canoj naknadi.
		\item Odluka se prosle\dj uje kod direktora na overu, \v cime se vr\v si Zaklju\v civanje re\v senja, i zatim u pisarnicu odakle se \v Salje po\v stom nezaposlenom licu.
	\end{itemize}
\end{frame}

\begin{frame}{Procena radne sposobnosti}
	\begin{itemize}
		\item Osoba, koja mo\v ze biti zaposlena ili nezaposlena, dolazi u Nacionalnu slu\v zbu za zapo\v sljavanje kako bi podnela zahtev za procenu radne sposobnosti nakon \v cega se formira komisija i vr\v si sam \v cin procene.
		\item Prvo se dolazi da Podnese zahtev za procenu radne sposobnosti.
		\item Onda podnosi potrebnu dokumentaciju.
		\item Zatim se zakazuje sastanak sa komisijom.
		\item Nakon sastanka sastavlja se re\v senje o radnoj sposobnosti osobe i ono se \v salje na ku\' cnu adresu. 
	\end{itemize}
\end{frame}

\begin{frame}{Administracija sistema}
	\begin{itemize}
		\item Administracija sistema odnosi se na dodavanje novih zaposlenih u Nacionalnoj slu\v zbi u sistem, odnosno pravljenje novih naloga, kako bi mogli da mu pristupaju. 
		\item Tako\dj e, podrazumeva brisanje korisnika iz sistema koji vi\v se ne rade u slu\v zbi. 
		\item Jo\v s jedan postupak koji se uklju\v cuje u ovaj slu\v caj upotrebe, a mo\v zda i najva\v zniji, jeste pravljenje backup-a baze podataka.
	\end{itemize}
\end{frame}

\section{Klase podataka}
\begin{frame}{Klase podataka}
	Analizom slu\v cajeva upotrebe, uo\v cene su slede\' ce grupe podataka:
\end{frame}

\subsection{Zaposleni u NSZ}
\begin{frame}{Zaposleni u NSZ}
	\begin{itemize}
		\item \v salterski slu\v zbenik
		\item savetnik
		\item upravni referent
		\item pravno lice
		\item direktor
	\end{itemize}

\end{frame}
	
\subsection{Nezaposlena lica na evidenciji}
\begin{frame}{Nezaposlena lica na evidenciji}
	\begin{itemize}
		\item  li\v cni podaci o nezaposlenom licu
		\item podaci o prethodnim zaposlenjima nezaposlenog lica
		\item podaci o javljanju
		\item podaci o radnoj sposobnosti
		\item podaci o nov\v canoj naknadi 
		\item podaci o nalogu za online javljanje
	\end{itemize}
\end{frame}	
	
\subsection{Poslodavci}
\begin{frame}{Poslodavci}
		\begin{itemize}
			\item podaci o kompaniji
			\item podaci o oglasima i konkursima za poslove
			\item podaci o potrebnim profilima ljudi
			\item podaci o nalogu za online otvaranje oglasa
		\end{itemize}
	\end{frame}	

\subsection{Zanimanja}
\begin{frame}{Zanimanja}
		\begin{itemize}
			\item podaci o zanimanjima
			\item podaci o stru\v cnoj spremi
		\end{itemize}
\end{frame}


\section{Zaklju\v cak}
\begin{frame}{Zaklju\v cak}
	\begin{itemize}
		\item Izrada ovog projekta je uklju\v cila modeliranje klju\v cnih delova sistema Nacionalne slu\v zbe za zapo\v sljavanje, ali ne i njihovu implementaciju. 
		\item Unapre\dj enje postoje\' ceg sistema je dato u vidu usluga putem interneta. \item Dalje unapre\dj enje bi podrazumevalo pravljenje alternativa postoje\' cim slu\v cajevima upotrebe koji bi se obavljali putem interneta, kao \v sto su: Prijava na evidenciju, Redovno javljanje, i Prijava za nov\v canu naknadu.

	\end{itemize}
\end{frame}	

\begin{frame}{Zaklju\v cak}
\begin{itemize}	
		\item Ovaj projekat se mo\v ze primeniti na bilo kakvu agenciju za zapo\v sljavanje.
		\item U razgovoru sa zaposlenima Nacionalne slu\v zbe zaklju\v cili smo da prostora za unapre\dj enje (olak\v savanje posla zaposlenima, ali i poslodavcima i nezaposlenim licima u ostvarivanju svojih ciljeva) ima jo\v s, ali neki od njih izlaze iz zakonskih okvira zbog poverljivosti informacija, stoga se njima nismo bavili u ovom projektu.
	\end{itemize}
\end{frame}

\begin{frame}
	Pitanja?
\end{frame}

\begin{frame}
	Hvala na pa\v znji!
\end{frame}

\end{document}